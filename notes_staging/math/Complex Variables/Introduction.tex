% title: Introduction
% date: 2026-01-08

$\mathbb{N}$: Natural Numbers = $\{0, 1, 2, \dots\}$

$\mathbb{Z}$: Integers = Natural numbers and their additive inverse

$\mathbb{Q}$: Rational numbers = $\{\frac{p}{q} | p, q \in \mathbb{Z}, q \neq 0\}$

$\mathbb{R}$: Real numbers = $\{\text{all points on the number line}\}$

It is also important to recall that $i^2 = -1$ 

\section*{Complex Numbers}
Written in the form $a+bi$ with $a, b \in \mathbb{R}$, a is the real part denoted by $\operatorname{R}(z)$. b is the imaginary part, denoted by $\operatorname{I}(z)$. The set of complex numbers can be defined as 
\[
\mathbb{C} = \{a + bi | a, b \in \mathbb{R}\}    
\]
Taking $z_1 = a + bi$ and $z_2 = c + di$, we can define

$z_1 + z_2 = (a+c)+(b+d)i$

$z_1 \times z_2 = (ac-bd)+(bc+ad)i$

\section*{Example}
Find all solutions to $z^2-z+1-i=0$

Strategy: write $z=a+bi$
\begin{itemize}
\item $I(z^2-z+1-i) = 0$
\item $R(z^2-z+1-i) = 0$
\end{itemize}

We can use completing the square ($(z^2+\frac{a}{2})^2 - \frac{a^2}{4}+b$)

$(z-\frac{1}{2})^2 +\frac{3}{4}-i$

Set $w = z-\frac{1}{2}$

$w^2 = -\frac{3}{4}+i$

Solving this gives us $a=\frac{1}{2},b=1 \ \vee \ a=\frac{-1}{2}, b=-1$